\chapter*{INTRODUCCIÓN}
\addcontentsline{toc}{chapter}{INTRODUCCIÓN}
\\

La minería de texto ha despertado un enorme interés en la comunidad científica, ya que ha aumentado su auge enormemente en los últimos años debido a la creciente cantidad de documentos disponibles en formato digital, y también a la creciente necesidad de organizar y obtener el conocimiento contenido en textos.
La literatura en el campo de la minería de textos ofrece una serie amplia de aplicaciones tales como la clasificación supervisada, la recuperación de información , la clasificación no supervisada (CNS) , extracción de entidades con nombre (NER), encontrar tendencias mediante nubes de palabras, extraer resúmenes de grandes volúmenes de texto, La mayoría de las técnicas propuestas en este ámbito se basan en el paradigma del aprendizaje artificial \cite{sebastiani2002machine}. Un pilar importante de la minería de textos es la representación de documentos no estructurados de tal forma que reflejen los distintos rasgos de su contenido de la mejor manera posible. Esto es sumamente importante cuando se trabaja con colecciones de documentos no etiquetados mediante una clase, como los que se trata en este proyecto denominados problemas de CNS \cite{jain2010data}.

%\textbf{Planteamiento del problema}
En el proceso investigativo realizado en el Grupo de Investigación Aplicada en Sistemas -  GRIAS (\textcolor{Cyan}{\underline{\url{http://grias.udenar.edu.co/grias/}}})
, en  la  línea  de  
investigación  de  Herramientas  y  Sistemas  de  Gestión  de Conocimiento  y  Recuperación  de  Información,
se  han  desarrollado  dos  proyectos  de investigación financiados por el sistema de investigaciones de la Universidad de Nariño: uno
por  la  convocatoria  estudiantil  denominado  ''Construcción  de  una  Ontología  de Aplicación  que  Soporte 
la  Búsqueda  Inteligente  sobre  los  Trabajos  de  Grado  de  la Universidad  de  Nariño  denominada  SAWA,  utilizando 
la  herramienta  de  software  libre Protégé'' \cite{cabrera2015swa} y otro en la convocatoria de trabajos de grado denominado
''UMAYUX: Un Modelo de Software de Gestión de Conocimiento Soportado en una Ontología Dinámica Débilmente Acoplado con un 
Gestor de Bases de Datos para la Universidad de Nariño'' \cite{benavides2014umayux}.  Estos  proyectos  fueron  delimitados  a  los  trabajos  de  grado 
del  programa  de Ingeniería de Sistemas de la Universidad de Nariño. Como resultado de estos proyectos se  cuenta  con  "MASKANA"  \cite{restrepo2015maskana}, 
un  prototipo  de  gestor  documental  para  recuperación  de información  relacionada  con 
los  trabajos  de  grado  del  programa  de  Ingeniería  de Sistemas  almacenados  en  formato  digital.  En estos proyectos se dispone 
de un repositorio textual de documentos no estructurados sin etiqueta de  clase, el cual se limito a encontrar relaciones semánticas
sin tener en cuenta los conceptos y NER.

Segun Vivas, Leticia y Coni, Ana Gar  \cite{VIVAS2013}, "los conceptos son elementos esenciales para el reconocimiento del mundo que nos rodea. Ellos constituyen una representación de una clase de cosas. Frecuentemente, se suelen confundir o utilizar indistintamente los términos concepto y palabra. El concepto [escuela],
por ejemplo, debe ser distinguido de la palabra 'escuela'. [Escuela] es un tipo de [institución educativa]. El concepto [escuela] puede,
por ejemplo, ser expresado por las palabras 'escuela', 'lugar para educar', 'institución educativa'. 
Los conceptos están profundamente relacionados unos con otros de manera que la activación de unos genera la 
activación de otros. Los vínculos que los interconectan se denominan relaciones conceptuales. Este tipo de relaciones no debe 
ser confundido con las relaciones entre términos o palabras. Mientras que a las primeras se las suele denominar relaciones conceptuales,
a las segundas se las suele denominar relaciones semánticas. Por ejemplo, las relaciones de sinonimia o de homonimia son
relaciones semánticas, mientras que las relaciones taxonómicas y 
temáticas son relaciones primordialmente entre conceptos."

Para Estes, Zachary , Golonka, Sabrina y Jones, Lara  \cite{estes2011thematic},  "las relaciones taxonómicas (también llamadas relaciones categoriales) son las que vinculan a conceptos de una misma categoría semántica.
Los objetos de la misma categoría taxonómica usualmente comparten el nombre genérico.
Dado que los componentes de este tipo de relaciones tienen rasgos comunes, las vinculaciones se establecen principalmente mediante mecanismos de detección de similitudes, es decir, comparando las propiedades de ambos conceptos.
Este tipo de relaciones permiten, agrupar los conceptos de una misma categoría, anticipar, mediante procesos de deducción e inferencia, las propiedades que tendrá un nuevo elemento que se incluya dentro de la categoría."

De acuerdo a Golonka, Sabrina y Estes, Zachary \cite{golonka2009thematic}, "las relaciones temáticas, Son relaciones contextuales entre objetos que no son del mismo tipo pero que pueden ser encontrados en los mismos esquemas. Específicamente, una cosa está temáticamente relacionada con otra cuando ambas desempeñan roles complementarios en el mismo escenario o situación." 

Segun Barsalou, Lawrence \cite{barsalou2003grounding}, " las relaciones temáticas permiten organizar contextualmente la experiencia así como establecer predicciones frente a situaciones futuras similares mediante el mecanismo de inferencia de completamiento de patrones." 

Se han propuesto diferentes técnicas de minería de textos, en \cite{troyano2003identificacion} 
describen los enfoques de extracción de NER y conceptos ligados al conocimiento, en \cite{figuerola2004algunas}
aplican técnicas para extraer palabras clave de documentos textuales.
En \cite{llorens1998caracteristicas}, \cite{santana2014aplicacion}, \cite{barrera2016mineria} y \cite{rodriguez2018metodos}
proponen aplicar técnicas de minería de textos para representar documentos no estructurados,
en \cite{MONTESYGOMEZ2005} y  \cite{munozutilizacion}
usan grafos conceptuales como representación del contenido de los textos, y obtiene algunos patrones descriptivos de los documentos aplicando varios tipos de operaciones sobre estos grafos.

Estos antecedentes proponen diferentes alternativas de minería de textos pero ninguno de ellos aplicado al dominio de trabajos de grado. 

Esto implico  investigar diferentes técnicas de minería de textos y minería de datos, aplicarlas
en el repositorio, evaluar su correcto funcionamiento e interpretar los patrones obtenidos generando conocimiento
útil para el repositorio de la biblioteca Alberto Quijano Guerrero de la universidad de Nariño.


\textbf{Objetivos}
 
Esta investigación tuvo como objetivo  descubrir relaciones conceptuales entre los trabajos de grado de la Universidad de Nariño (Colombia)
utilizando técnicas de minería de texto que facilite la recuperación de trabajos de grado
relacionados con la temática de la búsqueda identificando  similitudes y diferencias  entre ellos.

Para alcanzar este objetivo se plantearon los siguientes objetivos específicos 

\begin{itemize}
\item Apropiar el conocimiento en algoritmos de minería de texto, algoritmos de agrupación  y aprendizaje automático.
\item Construir, limpiar y transformar  el repositorio de documentos de trabajos de grado de la universidad de Nariño.
\item Implementar los algoritmos de minería de texto seleccionados en la herramienta MASKANA.
\item Descubrir las relaciones conceptuales entre los trabajos de grado y evaluar los resultados.
\end{itemize}


Este documento está organizado de la siguiente manera: en el capítulo 1 se elabora un marco teórico referente a minería de textos, en el capítulo 2 se presenta la metodología, en el capítulo 3 se interpretan y discuten los resultados y finalmente  en el capítulo 4 se muestra las conclusiones y trabajos futuros en base a los resultados.