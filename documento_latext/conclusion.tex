\chapter{ CONCLUSIONES Y TRABAJOS FUTUROS}

Con la culminación de esta tesis se logra  descubrir relaciones conceptuales entre los trabajos de grado de la Universidad de Nariño (Colombia) utilizando técnicas de minería de texto y facilitar la recuperación de trabajos de grado relacionados con temáticas de la búsquedas identificando similitudes y diferencias entre ellos.

Con la construcción, limpieza y transformación del repositorio de documentos de trabajos de grado de la universidad de Nariño; se logró estructurar este repositorio usando diferentes técnicas de minería de texto y se seleccionó la más adecuada que de acuerdo a los resultados fue Doc2vec.

Estructurar  el corpus de trabajos permito utilizar el algoritmo k-means, para encontrar relaciones categoriales y diferenciar los dominios de conocimiento del repositorio, los resultados mostraron que el número óptimo de categorías o grupos de conocimiento es 32.

Los modelos  Ner y  Word2vec permitieron interpretar el conocimiento relacionado a cada una de las categorías encontradas, estos 2 modelos fueron implementados en el algoritmo \ref{alg:maskanita2} el cual visualiza relaciones conceptuales temáticas  entre los documentos del repositorio. 

Las tareas de clasificación si bien sufren de sobre ajuste obtuvieron un buen desempeño, el modelo Xgboost alcanzó una medida de accuracy en el conjunto de testeo del 90\%, el cual resultó superior a los demás clasificadores, este modelo permite clasificar nuevos documentos que ingresen al repositorio y se implementó en el algoritmo \ref{alg:maskanita1};  para clasificar las temáticas de búsqueda en las categorías o grupos encontrados y con esto poder recuperar documentos similares a estas. 

Como trabajo a futuro se recomienda implementar los modelos y algoritmos realizados en esta tesis, en el domino de trabajos de investigación de la vicerrectoría de investigación de la universidad de Nariño.  
