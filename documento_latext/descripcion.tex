\chapter{Descripción del problema}


\section{Planteamiento del problema}

Se han propuesto algoritmos con el fin de descubrir patrones de agrupación analizando conjuntos de  datos con
el fin de encontrar los discos que se pueden unir entre instantes de tiempo consecutivos. El número de discos
posibles en un intervalo de tiempo dado puede ser bastante grande y el costo de unir los discos entre
los intervalos de tiempo puede llegar a ser muy alto. Manejar y analizar todas las combinaciones posibles tiene
un impacto directo en el rendimiento de los algoritmos. \cite{vieira2009line},  \cite{benkert2008reporting}
han probado algunas heurísticas y aproximaciones
encaminadas a reducir el número de discos evaluados.  \cite{romero2011mining} define una  metodología para
extraer patrones
de agrupación en conjuntos de datos de trayectoria basados en patrones frecuentes, el cual es un tema frecuentemente 
abordado en la minería de datos, con el objetivo de hacer frente a los inconvenientes antes mencionados. 

Sin  embargo,  el  número  y  la  calidad  de  los  patrones  descubiertos  hacen
particularmente difícil realizar una correcta interpretación de los resultados. Igualmente,
en \cite{vieira2009line} los resultados experimentales presentados en los anteriores trabajos muestran
que  todavía  existe  una  alta  complejidad  computacional  a  la  hora  de  efectuar  la
combinación  de  los  discos.  Esto  repercute  en  altos  tiempos  de  respuesta  y  un  alto
número  de  patrones  resultantes  que  hace  particularmente  dificil  su  interpretacion.
Inclusive, en \cite{romero2011mining} es necesario aplicar una etapa de postprocesamiento para descartar
patrones redundantes.

En  esta  investigación  se  propone  un  algoritmo  con  el  fin  de  encontrar  el  número  de
discos posibles en un intervalo de tiempo, mejorar la combinación de los mismos y por
ende su rendimiento sin perder la calidad del patrón encontrado. El algoritmo propuesto
será evaluado y comparado con los algoritmos propuestos en \cite{vieira2009line} y 
\cite{romero2011mining}.


\section{Objetivo general}

Proponer un algoritmo para el descubrimiento de patrones de agrupación de objetos móviles en bases 
de datos espacio temporales.

\section{Objetivos Específicos}

\begin{itemize}
 \item Apropiar el conocimiento en bases de datos espacio temporales, trayectorias y patrones de 
agrupamiento en objetos móviles.
 \item Implementar y evaluar los algoritmos seleccionados con el fin de identificar los problemas 
asociados a su rendimiento y su posterior comparación con la propuesta realizada.
 \item Diseñar el algoritmo para el descubrimiento de patrones de agrupación en objetos en 
movimiento.
 \item  Comparar los tiempos de respuesta y la claridad de los patrones obtenidos del algoritmo 
propuesto con respecto a los algoritmos existentes que fueron seleccionados.
\end{itemize}


\section{Organización del documento}

En el capítulo dos, se presenta la fundamentación teórica. En el capítulo tres, los trabajos 
realacionados sobre patrones de agrupamiento en objetos móviles. En el capítulo cuatro, se presenta
la metodología utilizada en la investigación. En el capítulo cinco, se describe la
alternativa propuesta en este proyecto denominada FPFlock con sus dos variaciones, una en línea y 
otra fuera de línea. En el capítulo seis, se presenta la experimentación computacional utilizando 
base de datos sintéticas y reales. En el capítulo siete, se enumeran los productos obtenidos a lo 
largo de la investigación. En el capítulo ocho, se muestra la discución sobre la experimentación 
computacional. En el capítulo nueve, se presentan las conclusiones y trabajos futuros, y por 
último se muestran las referencias bibliográficas y los anexos.  

