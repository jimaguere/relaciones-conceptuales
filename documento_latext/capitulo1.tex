%\chapter*{INTRODUCCIÓN}
%\chapter{INTRODUCCIÓN}
%\addcontentsline{toc}{chapter}{INTRODUCCIÓN}



\chapter{INTRODUCCIÓN}

\textbf{Planteamiento del problema}
En el proceso investigativo realizado en el Grupo de Investigación Aplicada en Sistemas -  GRIAS,  en  la  línea  de  
investigación  de  Herramientas  y  Sistemas  de  Gestión  de Conocimiento  y  Recuperación  de  Información,
se  han  desarrollado  dos  proyectos  de investigación financiados por el sistema de investigaciones de la Universidad de Nariño: uno
por  la  convocatoria  estudiantil  denominado  "Construcción  de  una  Ontología  de Aplicación  que  Soporte 
la  Búsqueda  Inteligente  sobre  los  Trabajos  de  Grado  de  la Universidad  de  Nariño  denominada  SAWA,  utilizando 
la  herramienta  de  software  libre Protégé" \cite{cabrera2015swa} y otro en la convocatoria de trabajos de grado denominado
"UMAYUX: Un Modelo de Software de Gestión de Conocimiento Soportado en una Ontología Dinámica Débilmente Acoplado con un 
Gestor de Bases de Datos para la Universidad de Nariño" \cite{restrepo2015maskana}.  Estos  proyectos  fueron  delimitados  a  los  trabajos  de  grado 
del  programa  de Ingeniería de Sistemas de la Universidad de Nariño. Como resultado de estos proyectos se  cuenta  con  MASKANA, 
un  prototipo  de  gestor  documental  para  recuperación  de información  relacionada  con 
los  trabajos  de  grado  del  programa  de  Ingeniería  de Sistemas  almacenados  en  formato  digital.  En estos proyectos se dispone 
de un repositorio textual de documentos no estructurados sin etiqueta de  clase, el cual se limito a encontrar relaciones semánticas
sin tener en cuenta los conceptos y entidades del conocimiento (NER).

"Los conceptos son elementos esenciales para el reconocimiento del mundo que nos rodea. Ellos constituyen una representación de una clase 
de cosas. Frecuentemente, se suelen confundir o utilizar indistintamente los términos concepto y palabra. El concepto [escuela],
por ejemplo, debe ser distinguido de la palabra 'escuela'. [Escuela] es un tipo de [institución educativa]. El concepto [escuela] puede,
por ejemplo, ser expresado por las palabras 'escuela', 'lugar para educar', 'institución educativa'. 
Los conceptos están profundamente relacionados unos con otros de manera que la activación de unos genera la 
activación de otros. Los vínculos que los interconectan se denominan relaciones conceptuales. Este tipo de relaciones no debe 
ser confundido con las relaciones entre términos o palabras. Mientras que a las primeras se las suele denominar relaciones conceptuales,
a las segundas se las suele denominar relaciones semánticas. Por ejemplo, las relaciones de sinonimia o de homonimia son
relaciones semánticas, mientras que las relaciones taxonómicas y 
temáticas son relaciones primordialmente entre conceptos" \cite{VIVAS2013}.

Se han propuesto diferentes técnicas de minería de textos, en \cite{troyano2003identificacion} 
describen los enfoques de extracción de NER y conceptos ligados al conocimiento, en \cite{figuerola2004algunas}
aplican técnicas para extraer palabras clave de documentos textuales.
En \cite{llorens1998caracteristicas}, \cite{santana2014aplicacion}, \cite{barrera2016mineria} y \cite{rodriguez2018metodos}
proponen aplicar técnicas de minería de textos para representar documentos no estructurados,
en \cite{MONTESYGOMEZ2005} y  \cite{munozutilizacion}
usan grafos conceptuales como representación del contenido de los textos, y obtiene algunos patrones descriptivos de los documentos aplicando varios tipos de operaciones sobre estos grafos.

Estos antecedentes proponen diferentes alternativas de minería de textos pero ninguno de ellos aplicado al dominio de trabajos de grado. 

Esto implico  investigar diferentes técnicas de minería de textos y minería de datos, aplicarlas
en el repositorio, evaluar su correcto funcionamiento e interpretar los patrones obtenidos generando conocimiento
útil para el repositorio de la biblioteca Alberto Quijano Guerrero de la universidad de Nariño.


\section{Trabajos Relacionados}

Las organizaciones mayormente disponen de información en documentos de texto no estructurado ,
se encuentran tipificados de esa manera ya que la información contenida en el documento no tiene ningún orden de estructura, ésta
información tiene mayor riesgo de no ser encontrada por los buscadores, pues no contiene parámetros establecidos que proporcionen
la información que se está buscando y sea presentada al usuario, por esta razón la minería de texto adquiere un rol importante
ya que es el proceso de extraer información interesante y conocimiento no trivial de textos no estructurados incluyendo tecnologías 
para extracción de información, seguimiento de temas, generación automática de resúmenes de textos, categorización,
agrupamiento , relaciones entre conceptos , visualización y respuesta automática de preguntas .  

\cite{troyano2003identificacion} describe los principales enfoques de extracción y reconocimiento de NER (entidades con nombre ) , 
el NER desempeña un papel muy importante en diversos problemas relacionados a la búsqueda automática y la categorización de textos.
 
En \cite{figuerola2004algunas} se propone un nuevo método para la caracterización de documentos que sin importar el idioma en el que el 
documento esté escrito, permite extraer el conjunto de palabras clave más adecuado. Su funcionamiento se basa en una Red Neuronal, que
luego de ser entrenada es capaz de decidir para cada término del documento si se trata de una palabra clave o no. El ingreso del documento a la Red Neuronal implicó la definición de una representación numérica adecuada que permite medir la participación de un término dentro del documento.

Utilizando  las técnicas de minería de texto se pretende obtener una serie de conjuntos de datos estructurados, 
para poder aplicar algoritmos de aprendizaje automático como se lo propone en \cite{llorens1998caracteristicas}, 
\cite{santana2014aplicacion}, \cite{barrera2016mineria} y \cite{rodriguez2018metodos}  logrando una categorización
y clasificación de documentos adecuada, \cite{ropero2014metodo}  propone un método que relaciona e integra técnicas de procesamiento de lenguaje natural,
agrupamiento (clustering) y modelos de Markov como una solución de bajo costo, dependiente del dominio, para la evaluación automática 
de la organización en textos argumentativos. 

Otra propuesta de utilización de minería de texto es la de Grobelnik, Mladenic  and Jermol  \cite{grobelnik2002exploiting}, en la cual se pretende potenciar una aplicación de construcción de ontologías/taxonomía a
partir de un conjunto de documentos planos, realizar búsquedas en la base de documentos y tratar problemas específicos del lenguaje,
por su parte \cite{cobo2006tecnicas},\cite{neto2000document} proponen sistemas  para resumir textos, agrupar documentos e interpretar el conocimiento de los grupos obtenidos para una  fácil compresión  por parte  del usuario.

Arco et al.\cite{arco2006agrupamiento} estudia el impacto de la representación del texto en el ámbito de la clasificación no supervisada (CNS) de documentos.
Tomando como referencia una representación basada en un modelo de espacio vectorial de términos, se analizan diferentes
técnicas de representación de los datos sobre espacios de menor dimensionalidad (obtenidas mediante técnicas de extracción de
términos como el Análisis de Semántica Latente, la Factorización en Matrices No Negativas y el Análisis en Componentes Independientes)
para mejorar la CNS de un corpus de documentos.


En \cite{MONTESYGOMEZ2005} y \cite{munozutilizacion} emplean minería de texto para la semejanza entre estructuras semánticas
usando grafos conceptuales como representación del contenido de los textos, y obtiene algunos patrones descriptivos de los documentos aplicando varios tipos de operaciones sobre estos grafos.



\textbf{Objetivos}
 

\textbf{Objetivo General}

Descubrir relaciones conceptuales entre los trabajos de grado de la Universidad de Nariño (Colombia)
utilizando técnicas de minería de texto que facilite la recuperación de trabajos de grado
relacionados con la temática de la búsqueda identificando  similitudes y diferencias  entre ellos.

\textbf{Objetivos específicos}
\begin{itemize}
\item Apropiar el conocimiento en algoritmos de minería de texto, algoritmos de agrupación  y aprendizaje automático.
\item Construir, limpiar y transformar  el repositorio de documentos de trabajos de grado de la universidad de Nariño.
\item Implementar los algoritmos de minería de texto seleccionados en la herramienta MASKANA.
\item Descubrir las relaciones conceptuales entre los trabajos de grado y evaluar los resultados.
\item Elaboración del documento final de tesis.
\end{itemize}

\textbf{Organización de la tesis}

En el capítulo 2 se elabora un marco teórico referente a minería de textos. 
En el capítulo 3 se presenta la construcción, limpieza, transformación del corpus de documentos de
trabajos de grado de la universidad de Nariño y los experimentos realizados. 
En el capítulo 4 se presentan los resultados.
En el capítulo 5 se muestra las conclusiones y trabajos futuros en base a los resultados.