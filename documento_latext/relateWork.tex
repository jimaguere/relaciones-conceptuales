\chapter{TRABAJOS RELACIONADOS}

Las capacidad de recolectar datos de objetos en movimiento ha ido aumentando 
rápidamente y el interés de consulta
de patrones que describen el comportamiento colectivo también ha aumentado. 
\cite{vieira2009line} enumera tres grupos de patrones
``colectivos'' en bases de datos de objetos en movimiento: clústers móviles, 
consulta de convoyes y patrones de agrupamiento.

Los clústers móviles \cite{jensen2007continuous} \cite{kalnis2005discovering} 
\cite{li2008mining} 
y  consultas de convoyes  \cite{jeung2008discovery-1} \cite{jeung2008convoy}, 
tienen en común que se basan en algoritmos
de clústering, principalmente en algoritmos basados en densidad como el 
algoritmo DBSCAN\cite{ester1996density}.

Los clústers móviles se definen entre dos instantes de tiempo consecutivos. Los 
clústers  se pueden unir sólo si el número de objetos comunes entre ellos están 
por encima del parámetro predefinido.
Un clúster es reportado si no hay otro nuevo clúster que pueda ser unido a éste. 
Este proceso se aplica cada vez para 
todos los instantes de tiempo en el conjunto de datos. 

Las consultas de convoyes se definen como un clúster denso de trayectorias que 
permanecen juntas al menos por un tiempo contínuo predefinido. 

Las principales diferencias entre las dos técnicas son la forma en que se unen 
los grupos  entre dos intervalos
consecutivos de tiempo y el uso de un parámetro adicional para especificar un 
tiempo mínimo de duración. Aunque
estos métodos están estrechamente relacionados con los patrones de agrupamiento, 
ninguno de ellos asume una  forma predefinida.

Previos trabajos de detección de patrones de agrupamiento móviles son descritos 
por \cite{ gudmundsson2006computing}
y \cite{benkert2008reporting}. Ellos introducen
el uso de discos con un radio predefinido para identificar  grupos de 
trayectorias que se mueven juntos en la misma
dirección, todas las trayectorias que se encuentran dentro del disco en un 
instante  de tiempo particular se 
considera un patrón candidato. La principal limitación de este proceso es que 
hay un número infinito de posibles
ubicaciones del disco en cualquier instante de tiempo. En efecto, en 
\cite{gudmundsson2006computing} se ha demostrado
que el descubrimiento de agrupaciones fijas, donde los patrones de las mismas 
entidades permanecen juntas durante 
todo el intervalo, es un problema NP-complejo. 

\cite{vieira2009line} son  los  primeros  en  presentar  una  solución  exacta  
para  reportar  patrones  de agrupación en tiempo polinomial, y también pueden trabajar efectivamente en 
tiempo real. Su trabajo revela que el tiempo de solución polinomial se puede encontrar 
a través de  la  identificación  de  un  número  discreto  de  ubicaciones  para  colocar 
 el  centro  del disco.  Los  autores  proponen  el  algoritmo  BFE  (Basic  Flock  Evaluation)  
basado  en  el tiempo de unión y combinación de los discos. La idea principal de este algoritmo 
es primero  encontrar  el  número  de  discos  válidos  en  cada  instante  de  
tiempo  y  luego combinarlos  uno  a  uno  entre  tiempos  adyacentes. Adicionalmente,  se  
proponen  otros cuatro  algoritmos  basados  en  métodos  heurísticos,  para  reducir  el  
número  total  de candidatos  a  ser  combinados  y,  por  lo  tanto,  el  costo  global del
algoritmo.  Sin  embargo,  el pseudocódigo y  los  resultados  experimentales
muestran todavía una alta complejidad computacional, largos tiempos de respuesta,
debido a que este algoritmo usa $\delta$ para partir los flocks hace que el número de patrones
encontrados sea mayor y esto hace difícil su interpretación.


\cite{romero2011mining} y \cite{turdu2014} proponen  una  metodología  que  
permite  identificar  patrones de agrupamiento utilizando tradicionales y potentes algoritmos de minería de datos 
usando patrones frecuentes, el cual fue comparado con BFE demostrando un alto 
rendimiento con conjuntos de datos sintéticos, aunque con conjuntos de datos reales el 
tiempo de respuesta siguió siendo eficiente pero similar a BFE. Este algoritmo trata el 
conjunto de trayectorias como una base de datos transaccional al convertir cada trayectoria, 
que se define  como  un  conjunto  de  lugares  visitados,  en  una  transacción,  
definida  como  un conjunto  de ítems.  De  esta  manera,  es  posible  aplicar  cualquier  
algoritmo  de  reglas  de asociación y encontrar patrones frecuentes sobre el conjunto dado, este algoritmo hace un llamado a LCM
propuesto por \cite{uno2004lcm} para encontrar patrones máximos y eso permite encontrar los flocks más largos.